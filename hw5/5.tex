\documentclass{article}
\usepackage{fontspec}
\usepackage{xcolor}

\usepackage{amsthm}
\usepackage{amsmath}
\usepackage{amssymb}
\usepackage{unicode-math}
\usepackage[makeroom]{cancel}

\usepackage[normalem]{ulem}

\setmainfont{Times New Roman}
\setmathfont{Latin Modern Math}

\setlength\parindent{0em}
\setlength\parskip{0.618em}
\usepackage[a4paper,lmargin=1in,rmargin=1in,tmargin=1in,bmargin=1in]{geometry}

\usepackage{enumitem}
\renewcommand\qed{$\blacksquare$}

\usepackage{mathabx,graphicx}
\usepackage{wrapfig}

\begin{document}

\begin{center}
  \textbf{MATH} 145B---\textbf{HOMEWORK} 4

  \color{red}R\color{teal}icardo
  \color{red}J\color{cyan}.
  \color{red}A\color{teal}cu$\color{red}{\widetilde{
      \color{teal}\text{n}}}$\color{teal}a\color{black}

  \color{teal}(\color{red}862079740\color{teal})\color{black}
\end{center}\vspace{1.618em}

\color{red}NOTE: All functions under discussion are considered
continuous, unless that's a property to be proved.\color{black}

\paragraph{1} Compute the fundamental group of the following spaces

(b) The cylinder, $S^1\times I$

$\pi_1(S^1\times I) =\pi_1(S^2)\times \pi_1(I)$

$I \sim \{*\} \implies \pi_1(D^2) = \pi_1(\{*\})= 1$ (the trivial
group)

And
$\pi_1(S^1) = \mathbb{Z}$

And, $\forall G:$ $G$ is a group, $1 \times G = G = G\times 1$

$\implies \pi_1(S^2\times I) = \mathbb{Z}$

(d) $X_1 := \{(x,y) \in \mathbb{R}^2|\enskip ‖(x,y)‖<1\}$

$X_1 \sim \{*\} \implies \pi_1(X_1) = \pi_1(\{*\})= 1$

(f) $X_2 := \{(x,y) \in \mathbb{R}^2|\enskip ‖(x,y)‖\geq 1\}$

$X_2 \sim S^1 \implies \pi_1(X_2) = \pi_1(S^1)= \mathbb{Z}$

(i) The subset of $\mathbb{R}^2$ given by $S^1\cup (\mathbb{R}^+\times
\mathbb{R})$.

$S^1\cup (\mathbb{R}^+\times
\mathbb{R})\sim S^1 \implies \pi_1(S^1\cup (\mathbb{R}^+\times
\mathbb{R})) = \pi_1(S^1)= \mathbb{Z}$

(j) $\mathbb{R}^3$ with the nonnegative $x$ and $y$ axes deleted.

$\mathbb{R}^3$ with the nonnegative $x$ and $y$ axes deleted is
homotopy equivalent to $S^1$. We can retract all the points to lie
arbitrarily close to the nonnegative $x$ and $y$ axes. This gives a
bent hollow cylinder, then we can retract along the walls of this
cylinder, until all the points lie about the origin. Then we can
expand the points around the origin to a circle of radius 1 that lies
at the $45$ degrees of the smallest angle that the nonnegative $x$ and
$y$ axes make with each other in the $(x,y)$-plane, measured from
either of the axes. So, it's
fundamental group is $\mathbb{Z}$.

\newpage
\paragraph{2} For each of the following spaces, show that the
fundamental group is isomorphic to thefundamental group of the figure
eight.

(a)  The torus $T^2$ $=S^1×S^1$ with a point removed.

(b) $\mathbb{R}^3$ with the nonnegative $x,y,$ and $z$ axes deleted.

(c) The subset of $\mathbb{R}^2$ given by $S^1\cup (\mathbb{R} \times \{0\} )$.

See the drawings.

\paragraph{3}  Let $A ⊂ \mathbb{R}^n$ be a subset, and $h:A→Y$ be a
continuous map with $h(a_0) =y_0$.  If $h$ extends to a continuous map
of $\mathbb{R}^n $, then the induced map $h_∗:\pi_1(A,a_0) →\pi_1(Y,y_0)$ is trivial.

pf.

$h$ extends to a continuous map of $\mathbb{R}^n$

$\implies \exits k: (\mathbb{R}^n, a_0)\rightarrow Y$: $k$ is
continuous and $k = h\circ i$, $i$ the inclusion map.

since $k_* = h_* \circ i_* \implies$ domain of $h_* =$  domain of
$i_* = \pi_1(\mathbb{R}^n, a_0)$

$\mathbb{R}^n, a_0 \sim \{*\} \implies \pi_1(\mathbb{R}^n, a_0) =
\pi_1(\{*\}) = 0$

$\implies i_*$ is trivial $\implies k_*$ is trivial $\implies h_*$ is
trivial

$\qed$

\paragraph{7} Calculate the homotopy groups of the complements $\mathbb{R}^2-S^1$
See the drawings.

\paragraph{9} Let $X$ be the union of two copies of $S^2$ with a
point in common.  What is the fundamental group of $X$?

See the drawings.

\end{document}

%%% Local Variables:
%%% mode: latex
%%% TeX-master: t
%%% End:
