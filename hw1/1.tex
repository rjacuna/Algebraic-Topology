\documentclass{article}
\usepackage{fontspec}
\usepackage{xcolor}

\usepackage{amsthm}
\usepackage{amsmath}
\usepackage{amssymb}
\usepackage{unicode-math}
\usepackage[makeroom]{cancel}

\usepackage[normalem]{ulem}

\setmainfont{Times New Roman}
\setmathfont{Latin Modern Math}

\setlength\parindent{0em}
\setlength\parskip{0.618em}
\usepackage[a4paper,lmargin=1in,rmargin=1in,tmargin=1in,bmargin=1in]{geometry}

\usepackage{enumitem}

\renewcommand\qedsymbol{$\blacksquare$}

\begin{document}

\begin{center}
  \textbf{MATH} 145B---\textbf{HOMEWORK} 1

  \color{red}R\color{teal}icardo
  \color{red}J\color{cyan}.
  \color{red}A\color{teal}cu$\color{red}{\widetilde{\color{teal}\text{n}}}$\color{teal}a\color{black}

  \color{teal}(\color{red}862079740\color{teal})\color{black}
\end{center}\vspace{1.618em}

$\guillemotright n \guillemotleft$ $:= $ Statement number $n$

$;$ $:= $ Reads `defined by' if preceded by a function type
declaration---i.e. $f: X \rightarrow Y; x \mapsto x^2$. reads $f$ from $X$
to $Y$ defined by $x$ maps to $x$ squared

$1_X$ $:=$ $\forall$ sets $X: X \neq \emptyset$, $1_X: X \rightarrow
X; x \mapsto x$ denotes the identity function on $X$

$(\_)^T$ $:=$ The transpose of $(\_)$

$[\_]_{m\times n}$ $:=$ $m \times n$ decorates the object and denotes
the dimension of the object. Used for clarity.

Crossley $:=$ ISBN 978-1-85233-782-7

|\_| $:=$ The cardinality of \_

$\equiv :=$ Constantly equals


\subsubsection*{6.1}

\paragraph{2}Show that the map $f : S^1 → S^1$ given by $f (x, y) =
(−x, −y)$ is homotopic to the
identity map.



\uwave{Pf\enskip.}
\vspace{0.618 em}
Want to show $f : S^1 \rightarrow S^1; (x,y) \mapsto (-x,-y)$ is
homotopic to $1_{S^1}$

$H:S^1\times[0,1]\rightarrow S^1; ((x,y),t)
\mapsto (\begin{pmatrix}\cos(\pi t)& -\sin(\pi t)\\ \sin(\pi t) & \cos(\pi t)\end{pmatrix}(x,y)^T)^T$

$H((x,y),0) = (\begin{pmatrix}\cos(0)& -\sin(0)\\ \sin(0)&
  \cos(0)\end{pmatrix}(x,y)^T)^T=([I]_{2\times2}(x,y)^T)^T = (x,y)$\\
$H((x,y),1) = (\begin{pmatrix}\cos(\pi)& -\sin(\pi)\\ \sin(\pi)&
  \cos(\pi)\end{pmatrix}(x,y)^T)^T= (-[I]_{2\times2}(x,y)^T)^T =
(-x,-y)$

This completes half of the proof, now let's consider the continuity of
$H$:\\

$(\begin{bmatrix}\cos(\pi t)& -\sin(\pi t)\\ \sin(\pi t) & \cos(\pi
  t)\end{bmatrix}_{2\times2}[(x,y)^T]_{2\times1})^T= \\

\begin{pmatrix}\cos(\pi
  t)x-\sin(\pi t)y\\ \sin(\pi t)x + \cos(\pi t)y\end{pmatrix}^T= \\

(\cos(\pi
t)x-\sin(\pi t)y, \sin(\pi t)x + \cos(\pi t)y) $

So H is continuous since both, $\pi_x\circ H$ and $\pi_y \circ H$ are
sums and products of continuous functions. This completes
$\frac{3}{4}$ of the proof.

Now, it just a matter of checking whether $H[S^1\times [0,1] ] \subset
S^1$

$(\cos(\pi t)x-\sin(\pi t)y)^2 + (\sin(\pi t)x + \cos(\pi t)y)^2=$\\
$\cos^2(\pi t)x^2-2\cos(\pi t)\sin(\pi t)xy+\sin^2(\pi
t)y^2+\sin^2(\pi t)x^2+2\cos(\pi t)\sin(\pi t)xy+\cos^2(\pi t)y^2=$\\
$\cos^2(\pi t)x^2-\cancel{2\cos(\pi t)\sin(\pi t)xy}+\sin^2(\pi
t)y^2+\sin^2(\pi t)x^2+\cancel{2\cos(\pi t)\sin(\pi t)xy}+\cos^2(\pi
t)y^2=$\\
$\cos^2(\pi t)x^2+\sin^2(\pi t)x^2+\sin^2(\pi t)y^2+\cos^2(\pi
t)y^2=$\\
$(\cos^2(\pi t)+\sin^2(\pi t))x^2+(\sin^2(\pi t)+\cos^2(\pi t))y^2=$\\
$1x^2+1y^2 = x^2 + y^2 =1 $, Since $(x,y) \in S^1$

So, $\forall (x,y)\in S^1: \forall t \in [0,1]: H((x,y),t)\in S^1$

\vspace{0.618 em}
$\blacksquare$
\newpage

\paragraph{3} Show that if $f, g : X → Y$ are homotopic and $h, k : Y → Z$ are homotopic, then $h ◦ f$
and $k ◦ g$ are homotopic.

\uwave{Pf\enskip.}
\vspace{0.618 em}

$f\sim g \implies \exists ($continuous $F: X\times[0,1]\rightarrow Y):
F(x,0)= f(x)$ and $F(x,1)= g(x)$

$h\sim k \implies k\sim h \implies \exists ($continuous $K: Y\times[0,1]\rightarrow Z):
K(x,0)= k(x)$ and $K(x,1)= h(x)$

Consider $H: X\times[0,1] \rightarrow Z; (x,t)\mapsto K(F(x,t),t)$\\
$\implies H(x,0) = K(F(x,0),0)= K(f(x),0) = h(f(x)) = h\circ f (x)$\\
and\\
$\implies H(x,1) = K(F(x,1),1)= K(g(x),1) = k(g(x)) = k\circ g (x)$

Since, both, $F$ and $K$ are continuous, $H$ is continuous as
$H$ is
the composition of two continuous functions.

So, $h\circ f \sim k\circ g$

\vspace{0.618 em}
$\blacksquare$


\paragraph{4} Given spaces $X$ and $Y$, let $[X, Y]$ denote the set of homotopy classes of maps of $X$
into $Y$. Let $I = [0, 1]$. Show that for any $X$, the set $[X, I]$ has a single element.

\uwave{Pf\enskip.}
\vspace{0.618 em}

Suppose, $[X,I] = \{f,g\} \Rightarrow f\not\sim g$ $\guillemotright 0 \guillemotleft$

$f \in [X,I] \implies \exists$ continuous $F: X\times I \rightarrow I$
where $(x,0) \mapsto f(x)$ and  $(x,1) \mapsto 1$

$g \in [X,I] \implies \exists$ continuous $G: X\times I \rightarrow I$
where$
(x,0) \mapsto g(x)$ and  $(x,1) \mapsto 1$

Consider, $H: X\times I\rightarrow I; H(x,t) = \begin{cases}
  F(x,2t), t\leq 1/2\\
  G(x,2-2t), t \geq 1/2 \end{cases}$

$\Rightarrow H(x,0) = F(x,0) = f(x)$\\
$\Rightarrow H(x,1) = G(x,2-2) = G(x,0) = f(x)$

Since, $H$ equals $F$ and $G$ exclusively $\forall t \in [0,\frac{1}{2})$ and
$\forall t \in (\frac{1}{2}, 1]$ respectively. \\
$H$ is automatically continuous on
$[0,\frac{1}{2})\cup(\frac{1}{2},1]$

The gluing lemma tells us it is enough to check that
$H$ is continuous at $\frac{1}{2}$ is well defined---i.e
check
\[F(x,2\frac{1}{2}) \stackrel{?}{=} G(x,2-2\frac{1}{2})\]

Indeed it is, since
$F(x,2\frac{1}{2}) = F(x,1) = 1 = G(x,1) = G(x, 2 - 1) = G(x,2-2\frac{1}{2})$

$\Rightarrow$ $H$ is a homotopy from $f$ to $g$ $\Rightarrow f \sim g$ which contradicts $\guillemotright 0
\guillemotleft$

So, $[X,I]$ has one element.

\vspace{0.618 em}
$\blacksquare$
\newpage
\paragraph{11} Show that if $g : S^2 → S^2$ is continuous and $g(x) \neq g(−x)$ for all $x ∈ S^2$ , then $g$ is
surjective. (Hint: If $p ∈ S^2$ , then $S^2 \backslash \{p\}$ is homeomorphic to $\mathbb{R}^2$ . Then use Borsuk-Ulam Theorem)

\uwave{Pf\enskip.}
\vspace{0.618 em}
WTS $g : S^2 → S^2$ is continuous and $g(x) \neq g(−x)$ for all $x ∈ S^2$ $\implies$ $g$ is
surjective.

Suppose $\exists p\in S^2: \forall x \in S^2:
g(x)\neq p$ $\guillemotright 0 \guillemotleft$

$S^2\backslash\{p\}$ is homeomorphic to
$\mathbb{R}^2$ (by Hint)
$\Rightarrow\\ \exists f: S^2\backslash\{p\} \rightarrow \mathbb{R}^2$
and $\exists f^{-1}: \mathbb{R}^2\rightarrow  S^2\backslash\{p\}$:\\ $f\circ
f^{-1} = 1_{\mathbb{R}^2}$ and $f^{-1}\circ f =
1_{S^2\backslash\{p\}}$ $\guillemotright 1 \guillemotleft$

Induce $g$ into $S^2\backslash\{p\}$ with $g': S^2\rightarrow
S^2\backslash\{p\}; x \mapsto g(x)$

Consider $f\circ g':S^2\rightarrow  \mathbb{R}^2$.
$f\circ g'$ is continuous since it is a composition of continuous functions.\\
$\exists x \in S^2: f\circ g'(x)=f\circ g'(-x)$ (By Borsuk-Ulam)\\
$\Rightarrow f(g'(x))=f(g'(-x))$ (by definition of $\circ$)\\
$\Rightarrow f(g(x))=f(g(-x))$ (by definition of $g'$)\\
$\Rightarrow f^{-1}(f(g(x)))=f^{-1}(f(g(-x)))$ (by applying $f^{-1}$)\\
$\Rightarrow f^{-1}\circ f(g(x))=f^{-1}\circ f(g(-x))$ (by definition of $\circ$)\\
$\Rightarrow 1_{S^2\backslash\{p\}}(g(x))=1_{S^2\backslash\{p\}}(g(-x))$ (by $\guillemotright 1 \guillemotleft$)\\
$\Rightarrow g(x)=g(-x)$

This contradicts the property that $\forall x\in S^2: g(x)\neq g(-x)$
so $\guillemotright 0 \guillemotleft$ is false.

$\Rightarrow \exists p \in S^2: \forall x \in S^2: g(x)=p$\\
$\Rightarrow g[S^2]=S^2$\\
$\Rightarrow g$ is surjective

\vspace{0.618 em}
$\blacksquare$

\newpage
\subsubsection*{6.2}

\paragraph{2} Write down a homotopy equivalence between $(0, 1)$ and $[0, 1]$.

\uwave{Pf\enskip.}
\vspace{0.618 em}

$f:(0,1) \rightarrow [0,1]; x \mapsto 1/2$\\
$g:[0,1] \rightarrow (0,1); y \mapsto 1/2$\\

Since,
$[0,1]\stackrel{g}{\rightarrow}(0,1)\stackrel{f}{\rightarrow}[0,1]$
and
$f\circ g(y) = f(g(y))=f(1/2)=1/2$\\
All, continuous functions, on $[0,1]$ are homotopic to constant maps.\\
In particular $1_{[0,1]} \sim 1/2 = f \circ g$\\
Since,
$(0,1)\stackrel{f}{\rightarrow}[0,1]\stackrel{g}{\rightarrow}(0,1)$
and
$g\circ f(x) = g(f(x))=g(1/2)=1/2$\\
Consider, $H: (0,1)\times [0,1] \rightarrow (0,1); (x,t) \mapsto t/2
+(1-t)x$\\
$H$ is continuous since it is a polynomial.\\
$H(x,0) = x$ $\Rightarrow $ at $0, H  = 1_{(0,1)}$\\
And since $H(x,1) = 1/2$ $\Rightarrow$ $1_{(0,1)} \sim 1/2 = g\circ f$\\
So $(0,1) \simeq [0,1]$

\vspace{0.618 em}
$\blacksquare$

\newpage
\paragraph{3} Show that a space $X$ is contractible iff every map $f : X → Y$, for arbitrary $Y$ ,
is nullhomotopic. Similarly, show $X$ is contractible iff every map $f : Y → X$ is
nullhomotopic.

The answer is in two parts 0 and 1

\uwave{Pf\enskip. 0}
\vspace{0.618 em}

WTS $X$ is contractible iff every map $f : X → Y$, for arbitrary $Y$ ,
is nullhomotopic.

$(\Leftarrow)$ Ass. every map $f : X → Y$, for arbitrary $Y$,
is nullhomotopic.

$\Rightarrow$ $\forall($$f_j: X\rightarrow Y): \exists y_i\in Y: \exists (c_i: X \rightarrow
Y;x\mapsto y_i): f_j\sim c_i$\\
$\Rightarrow$ $[X,Y] = \{c_i:X\rightarrow Y\}$\\
$\Rightarrow$ $[X,Y]$ has at most $|Y|$ elements

Since $Y$ is arbitrary, we can choose $Y = \{0\}$.\\
$\Rightarrow$ $[X,\{0\}]$ has 1 element.\\
$\Rightarrow$ $[X,\{0\}] = \{c: X \rightarrow \{0\}; x \mapsto 0\}$.

Consider $g: \{0\}\rightarrow X; 0 \mapsto x_0$\\
$c\circ g (0) = c(g(0))= c(x_0) = 0$ $\Rightarrow$ $c\circ g =
1_{\{0\}}$\\
$g\circ c (x) = g(c(x))= g(0) = x_0$ $\Rightarrow$ $g\circ c \equiv
x_0$\\

Now, since $Y$ is arbitrary, we can choose again $Y = X$.\\
$\Rightarrow$ $[X,X] = \{c_i: X\rightarrow X; x\mapsto  x_i| x_i \in
X\}$.

So, for some $x_0 \in X$, $g\circ c:X\rightarrow X \equiv x_0 \equiv c_0 \sim 1_X$

$\Rightarrow$ $X\simeq \{0\}$ $\Rightarrow$ $X$ is contractible

$(\Rightarrow)$ Ass. $X$ is contractible

$\Rightarrow$ $X \simeq \{0\}$\\
Let, $Y$ be an arbitrary topological space.\\
Then, by Lemma $6.10$ in Crossley $[X,Y] = [\{0\}, Y]$

Since,$\{0\}$ has one element,  $\forall (g:\{0\}\rightarrow Y),
g$ has to be a constant map.

$[\{0\},Y] = \{c_i: \{0\}\rightarrow
Y; 0\mapsto y_i| y_i \in Y\}$ and $[X,Y] = [\{0\},Y]$ \\$\Rightarrow$\\ $[X,Y] =
\{k_i:X \rightarrow Y; x\mapsto y_i| y_i\in Y\}$\\
$\Rightarrow$\\
$\forall$$(f: X\rightarrow Y): \exists y_i \in Y: (k_i: X \rightarrow
Y;x\mapsto y_i): f\sim k_i$\\
So, all maps $f$ from $X$ to $Y$ are nullhomotopic

So, $X$ is contractible iff every map $f : X → Y$, for arbitrary $Y$ ,
is nullhomotopic.

\vspace{0.618 em}
$\blacksquare$


\newpage
\uwave{Pf\enskip. 1}
\vspace{0.618 em}

Also WTS $X$ is contractible iff every map $f : Y → X$ is
nullhomotopic.

$(\Leftarrow)$ Ass. every map $f : Y → X$ is
nullhomotopic.

$\Rightarrow$ $\forall($$f_j: Y\rightarrow X):\exists x_i \in X: \exists (m_i: Y \rightarrow
X; y \mapsto x_i): f_j\sim m_i$\\
$\Rightarrow$ $[Y,X]$ has at most $|X|$ elements

Since, $Y$ is arbitrary, we can choose $Y=\{x_0\}$: $x_0 \in X$.\\
Now, since $\{x_0\}$ has one element $\forall (g_i: \{x_0\}\rightarrow X)$,
$g_i$ is constant.\\
So, $[\{x_0\},X] = \{g_i:\{x_0\} \rightarrow X; x_0 \mapsto x_i| x_i\in X\}$

We can choose again, $Y = X$: $[X,X] = \{m_i: X \rightarrow X;
x\mapsto x_i\}$

$\Rightarrow$ $\exists x_0 \in X$: $m_0 \sim 1_X$

Consider, $l: X \rightarrow \{x_0\}; x \mapsto x_0$\\
$l \circ g_0 (x_0)= l(g_0(x_0)) = l(x_0) = x_0 \Rightarrow l\circ g_0
= 1_{\{x_0\}}$

$g_0\circ l (x) = g_0(l(x)) = g_0(x_0) = x_0$ $\Rightarrow$ $g_0\circ
l = m_0 \sim 1_X$

So, $X\simeq \{x_0\}$ $\Rightarrow X$ is contractible.

$(\Rightarrow)$ Ass. $X$ is contractible.

$\Rightarrow$ $X \simeq \{x_0\}$\\
Let, $Y$ be an arbitrary topological space.\\
Then, by Lemma $6.10$ in Crossley $[Y,X] = [Y,\{x_0\}]$

Since, $\{x_0\}$ has one element, $[Y,\{x_0\}] = \{c: Y\rightarrow \{x_0\};
y\mapsto x_0\}$ also has one element $c$.\\
And, since $[Y,X] = [Y, \{x_0\}]$ by the lemma.

$\forall (f:X \rightarrow Y): f\sim c$

So, all functions from $X$ to $Y$ are nullhomotopic.

So, $X$ is contractible iff every map $f : Y → X$ is
nullhomotopic.

\vspace{0.618 em}
$\blacksquare$


\end{document}

%%% Local Variables:
%%% mode: latex
%%% TeX-master: t
%%% End:
