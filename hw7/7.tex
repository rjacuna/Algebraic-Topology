\documentclass{article}
\usepackage{fontspec}
\usepackage{xcolor}

\usepackage{amsthm}
\usepackage{amsmath}
\usepackage{amssymb}
\usepackage{unicode-math}
\usepackage[makeroom]{cancel}

\usepackage[normalem]{ulem}

\setmainfont{Times New Roman}
\setmathfont{Latin Modern Math}

\setlength\parindent{0em}
\setlength\parskip{0.618em}
\usepackage[a4paper,lmargin=1in,rmargin=1in,tmargin=1in,bmargin=1in]{geometry}

\usepackage{enumitem}
\renewcommand\qed{$\blacksquare$}

\usepackage{mathabx,graphicx}
\usepackage{wrapfig}

\begin{document}

\begin{center}
  \textbf{MATH} 145B---\textbf{HOMEWORK} 7

  \color{red}R\color{teal}icardo
  \color{red}J\color{cyan}.
  \color{red}A\color{teal}cu$\color{red}{\widetilde{
      \color{teal}\text{n}}}$\color{teal}a\color{black}

  \color{teal}(\color{red}862079740\color{teal})\color{black}
\end{center}\vspace{1.618em}


\paragraph{1. (a)}

Reading from the picture from Homework 6:

$C_1(S^1) = \{\lambda_0e_0+\lambda_1e_1+\lambda_1e_2|
\lambda_i\in\mathbb{Z}\}$

$
C_0(S^1) = \{\lambda_0v_0+ \lambda_1v_1+\lambda_1v_2| \lambda_i\in\mathbb{Z}\} $

$C_n(S^1) = 0$, $\forall n \geq 2$


$\forall n \in \mathbb{N}: n\geq 2 \implies \bold{Ker}\enskip
\partial_n =  \bold{Im}\enskip \partial_n = 0$

$\implies \forall n \in \mathbb{N}: n\geq 2: H_n(S^1) = \frac{0}{0} = 0$

$\partial_1(e_0) = v_2 - v_0 $

$\partial_1(e_1) = v_1 - v_2 $

$\partial_1(e_2) = v_0 - v_1 $

$\implies \bold{Im}\enskip{\partial_1} = <v_2-v_0, v_1 -v_2, v_0 -v_1>$

$v_0-v_1 = -(v_2-v_0+v_1-v_2)
\implies$ one of the generators is
redundant

$\implies \bold{Im}\enskip{\partial_1} = <v_2-v_0, v_1 -v_2>$

Solve $\partial_1(\lambda_0e_0+ \lambda_1e_1+ \lambda_2e_2) = \lambda_0(v_2 -
v_0) +\lambda_1(v_1 - v_2) +
\lambda_2(v_0 - v_1) = 0$

$\implies \lambda_0(v_2 -
v_0) +\lambda_1(v_1 - v_2) +
\lambda_2(v_0 -v_1) =
(\lambda_2 -\lambda_0)v_0
+(\lambda_1-\lambda_2)v_1
+ (\lambda_0 -\lambda_1)v_2 = 0$

$\implies
\lambda_2 -\lambda_0 = 0$ and $
\lambda_1-\lambda_2 = 0$ and
$ \lambda_0 -\lambda_1 =0 \implies
\lambda_2 = \lambda_0$ and
$\lambda_1 = \lambda_2$ and
$\lambda_0 = \lambda_1$

$\implies \lambda_0 = \lambda_1 =
\lambda_2 := \lambda$

$\implies \partial_1(\lambda(e_0+e_1+e_2)) = 0 $

$\implies \bold{Ker}\enskip{\partial_1} = <e_0+e_1+e_2>$



$\implies H_1(S^1;\mathbb{Z}_2) =$
$\frac{\bold{Ker}\enskip \partial_1}{\bold{Im}\enskip\partial_{2}} =$
$\frac{<e_0+e_1+e_2>}{0} = \mathbb{Z}_2$

$\partial_0(v_0) = \partial_0(v_1) = \partial_0(v_2) = 0$

$\implies \bold{Im}\enskip \partial_0 = 0$ and $\bold{Ker}\enskip \partial_0 = <v_0,v_1,v_2>$

$\implies H_0(S^1;\mathbb{Z}_2) =$
$\frac{\bold{Ker}\partial_0}{\bold{Im}\partial_{1}} =$
$\frac{<v_0,v_1,v_2>}{<v_2-v_0,v_1-v_2>}$

One can change the basis of $C_0(S^1) = \bold{Ker} \partial_0$ as
such: $<v_0,v_1,v_2> = <v_0,v_1-v_2, v_2-v_0>$

$\implies H_0(S^1;\mathbb{Z}_2) = \frac{<v_0,v_1-v_2,
  v_2-v_0>}{<v_2-v_0,v_1-v_2>} =  <v_0> = \mathbb{Z}_2$

So,

$H_n(Annulus;\mathbb{Z}_2) = \begin{cases} \mathbb{Z}_2$ , $ n = \in \{0, 1\}\\ 0$ ,
  else$\end{cases}$

(c) By The Universal Coefficient Theorem for $\mathbb{Z}_2$:

$\forall n \in \mathbb{N}: n\geq 3:$

$H_n(Annulus;\mathbb{Z}_2) = (H_n(Annulus)\otimes \mathbb{Z}_2)\oplus
$Tor$(H_{n-1}(Annulus), \mathbb{Z}_2) = (0 \otimes \mathbb{Z}_2)\oplus
$Tor$(0, \mathbb{Z}_2) = 0 \oplus 0 = 0 $


$H_2(Annulus;\mathbb{Z}_2) = (H_2(Annulus)\otimes \mathbb{Z}_2)\oplus
$Tor$(H_1(Annulus), \mathbb{Z}_2) = (0 \otimes \mathbb{Z}_2)\oplus
$Tor$(\mathbb{Z}, \mathbb{Z}_2) = 0 \oplus 0 = 0 $


$H_1(Annulus;\mathbb{Z}_2) = (H_1(Annulus)\otimes \mathbb{Z}_2)\oplus
$Tor$(H_0(Annulus), \mathbb{Z}_2) = (\mathbb{Z}\otimes \mathbb{Z}_2)\oplus
$Tor$(\mathbb{Z}, \mathbb{Z}_2) = \mathbb{Z}_2\oplus 0 = \mathbb{Z}_2 $

$H_0(Annulus;\mathbb{Z}_2) = (H_{0}(Annulus)\otimes
\mathbb{Z}_2)\oplus $Tor$(H_{-1}(Annulus), \mathbb{Z}_2) = (\mathbb{Z}\otimes
\mathbb{Z}_2)\oplus $Tor$(0, \mathbb{Z}_2) = \mathbb{Z}_2\oplus 0 = \mathbb{Z}_2$

\newpage
\paragraph{5. (a)}


$\forall n\in \mathbb{N}: n\geq 3
\implies C_n(S^2) = 0$

$C_2(S^2) = \{\lambda f| \lambda \in
\mathbb{Z}\}$

$C_1(S^2) = 0$,

$C_0(S^2) = \{\lambda v| \lambda \in \mathbb{Z}\}$

$\forall n \in \mathbb{N}: n\geq 3
\implies$ $\bold{Im}$ $\partial_n = 0 =
\bold{Ker}$ $\partial_n$. Because they're all $0$.

$f$ is the only element of $C_2(S^2)
\implies$
$\partial_2(f) = 0 \implies \bold{Im}$ $\partial_2 = 0$ and
$\bold{Ker}$ $\partial_2 = <f>$

$\implies H_2(S^2;\mathbb{Z}_2) =
\frac{\bold{Ker}\partial_2}{\bold{Im}\partial_{3}} = \frac{<f>}{0} =
<f> = \mathbb{Z}_2$

$0$ is the only element of $C_1(S^2)$ $\implies \partial_1(0) = 0 \implies \bold{Im}$ $\partial_1 = 0$ and
$\bold{Ker}$ $\partial_1 = 0$

$v$ is the only element of $C_0(S^2) \implies \partial_0(v) = 0 \implies \bold{Im}$ $\partial_0 = 0$ and
$\bold{Ker}$ $\partial_0 = <v>$

$\implies H_0(S^2;\mathbb{Z}_2) =
\frac{\bold{Ker}\partial_0}{\bold{Im}\partial_{2}} = \frac{<v>}{0} =
<v> = \mathbb{Z}_2$


So,

$H_n(S^2;\mathbb{Z}_2) = \begin{cases} \mathbb{Z}_2$ , $ n = \in \{0, 2\}\\ 0$ ,
  else$\end{cases}$



(c) By The Universal Coefficient Theorem for $\mathbb{Z}_2$:

$\forall n \in \mathbb{N}: n\geq 4:$

$H_n(S^2;\mathbb{Z}_2) = (H_n(S^2)\otimes \mathbb{Z}_2)\oplus
$Tor$(H_{n-1}(S^2), \mathbb{Z}_2) = (0 \otimes \mathbb{Z}_2)\oplus
$Tor$(0, \mathbb{Z}_2) = 0 \oplus 0 = 0 $

$H_3(S^2;\mathbb{Z}_2) = (H_3(S^2)\otimes \mathbb{Z}_2)\oplus
$Tor$(H_2(S^2), \mathbb{Z}_2) = (0 \otimes \mathbb{Z}_2)\oplus
$Tor$(\mathbb{Z}, \mathbb{Z}_2) = 0 \oplus 0 = 0 $


$H_2(S^2;\mathbb{Z}_2) = (H_2(S^2)\otimes \mathbb{Z}_2)\oplus
$Tor$(H_1(S^2), \mathbb{Z}_2) = (\mathbb{Z} \otimes \mathbb{Z}_2)\oplus
$Tor$(0, \mathbb{Z}_2) = \mathbb{Z}_2 \oplus 0 = \mathbb{Z}_2 $


$H_1(S^2;\mathbb{Z}_2) = (H_1(S^2)\otimes \mathbb{Z}_2)\oplus
$Tor$(H_0(S^2), \mathbb{Z}_2) = (0\otimes \mathbb{Z}_2)\oplus
$Tor$(\mathbb{Z}, \mathbb{Z}_2) = 0\oplus 0 = 0$

$H_0(S^2;\mathbb{Z}_2) = (H_{0}(S^2)\otimes
\mathbb{Z}_2)\oplus $Tor$(H_{-1}(S^2), \mathbb{Z}_2) = (\mathbb{Z}\otimes
\mathbb{Z}_2)\oplus $Tor$(0, \mathbb{Z}_2) = \mathbb{Z}_2\oplus 0 = \mathbb{Z}_2$


\newpage
\paragraph{7. (a)}


$C_2(S^2)
= \{\lambda f| \lambda \in \mathbb{Z}\}$


$C_1(\mathbb{R}P^2) = \{\lambda_0a+\lambda_1b| \lambda_i
\in \mathbb{Z}\}$

$C_0(\mathbb{R}P^2) = \{\lambda_0v_0+\lambda_1v_1| \lambda_i \in \mathbb{Z}\}$

$\forall n\in \mathbb{N}: n\geq 3 \implies C_n(\mathbb{R}P^2) = 0$
$\implies $ $\bold{Im}$ $\partial_n = 0 =
\bold{Ker}$ $\partial_n$

$\implies H_n(\mathbb{R}P^2) =
\frac{\bold{Ker}
  \partial_n}{\bold{Im}
  \partial_{n+1}} = \frac{0}{0} =
0$, for $n\geq 3$


$f$ is the only element of $C_2(\mathbb{R}P^2) \implies \partial_2(f)
= 2(b-a) \equiv 0$ $($mod $2)$

$\implies \bold{Im}$ $\partial_2 = 0$

$\implies \bold{Ker}$ $\partial_2 = <f>$

$\implies H_2(\mathbb{R}P^2) =
\frac{\bold{Ker}
  \partial_2}{\bold{Im}
  \partial_{3}} = \frac{<f>}{0} =
\mathbb{Z}_2$


$\partial_1(a) = v-w$ and
$\partial_1(b) = v-w$

$\implies \bold{Im}$ $\partial_1 = <v-w>$

$y_1(v-w) +y_2(v-w) = (y_1+y_2)v -
(y_1+y_2)w = 0 \implies y_1+y_2 = 0
\implies -y_1 = y_2 $

$\implies \bold{Ker}$ $\partial_1 =
<b-a>$


$\implies H_1(\mathbb{R}P^2) =
\frac{\bold{Ker}
  \partial_1}{\bold{Im} \partial_2}
= \frac{<b-a>}{0} =
\frac{\mathbb{Z}_2}{0} = \mathbb{Z}_2$

$\partial_0(v_0) = \partial_0(v_1) = 0$ $\implies \bold{Im}$
$\partial_0 = 0$ and $\bold{Ker}$ $\partial_0 = <v_0,v_1>$

One can change the generators $<v_0,v_1>$ to $<v_1-v_0,v_0>$.

$\implies H_0(\mathbb{R}P^2) = \frac{\bold{Ker} \partial_0}{\bold{Im}
  \partial_1} = \frac{<v_0,v_1>}{<v_1-v_0>} =
\frac{<v_1-v_0,v_0>}{<v_1-v_0>} = <v_0> = \mathbb{Z}_2$

So,

$H_n(\mathbb{R}P^2)
= \begin{cases}\mathbb{Z}_2, n \in\{0,1,2\}\\
  0,$ else$
\end{cases}$

(c) By The Universal Coefficient Theorem for $\mathbb{Z}_2$:

$\forall n \in \mathbb{N}: n\geq 3:$

$H_n(\mathbb{R}P^2;\mathbb{Z}_2) = (H_n(\mathbb{R}P^2)\otimes \mathbb{Z}_2)\oplus
$Tor$(H_{n-1}(\mathbb{R}P^2), \mathbb{Z}_2) = (0 \otimes \mathbb{Z}_2)\oplus
$Tor$(0, \mathbb{Z}_2) = 0 \oplus 0 = 0 $

$H_2(\mathbb{R}P^2;\mathbb{Z}_2) = (H_2(\mathbb{R}P^2)\otimes \mathbb{Z}_2)\oplus
$Tor$(H_1(\mathbb{R}P^2), \mathbb{Z}_2) = (0 \otimes \mathbb{Z}_2)\oplus
$Tor$(\mathbb{Z}_2, \mathbb{Z}_2) = 0\oplus \mathbb{Z}_2 = \mathbb{Z}_2 $


$H_1(\mathbb{R}P^2;\mathbb{Z}_2) = (H_1(\mathbb{R}P^2)\otimes \mathbb{Z}_2)\oplus
$Tor$(H_0(\mathbb{R}P^2), \mathbb{Z}_2) = (\mathbb{Z}_2\otimes \mathbb{Z}_2)\oplus
$Tor$(\mathbb{Z}, \mathbb{Z}_2) = \mathbb{Z}_2\oplus 0 = \mathbb{Z}_2$

$H_0(\mathbb{R}P^2;\mathbb{Z}_2) = (H_{0}(\mathbb{R}P^2)\otimes
\mathbb{Z}_2)\oplus $Tor$(H_{-1}(\mathbb{R}P^2), \mathbb{Z}_2) = (\mathbb{Z}\otimes
\mathbb{Z}_2)\oplus $Tor$(0, \mathbb{Z}_2) = \mathbb{Z}_2\oplus 0 = \mathbb{Z}_2$

\newpage
\paragraph{11} See new drawing.

$C_3(T^3) = \{\lambda U |$ $\lambda \in\mathbb{Z}\}$

$C_2(T^3) = \{\lambda_1f_1 +\lambda_2f_2 +\lambda_3f_3|$ $\lambda_i\in\mathbb{Z}\}$

$C_1(T^3) = \{\lambda_0a+\lambda_1b +\lambda_2c|$ $\lambda_i\in\mathbb{Z}\}$

$C_0(T^3) = \{\lambda v |$ $\lambda\in\mathbb{Z}\}$

$\forall n\in \mathbb{N}: n\geq 4 \implies C_n(T^3) = 0$
$\implies $ $\bold{Im}$ $\partial_n = 0 =
\bold{Ker}$ $\partial_n$

$\implies H_n(T^3) =
\frac{\bold{Ker}
  \partial_n}{\bold{Im}
  \partial_{n+1}} = \frac{0}{0} =
0$, for $n\geq 4$

$\partial(U) = 0$

$\implies \bold{Im}$ $\partial_3 = 0$
$\implies \bold{Ker}$ $\partial_3 = <U>$

$\implies H_3(T^3) =
\frac{\bold{Ker}\enskip\partial_3}{\bold{Im}\enskip\partial_4} =
\frac{<U>}{0} = <U> = \mathbb{Z}$

$\partial_2(f_1) = a-b-a+b = 0$

$\partial_2(f_2) = a-c-a+c = 0$

$\partial_2(f_3) = c-b-c+b = 0$


$\implies \bold{Ker}$ $\partial_2 = <f_1,f_2,f_3>$

$\implies \bold{Im}$ $\partial_2 = 0$


$\implies H_2(T^3) =
\frac{\bold{Ker}\enskip\partial_2}{\bold{Im}\enskip\partial_3} =
\frac{<f_1,f_2,f_3>}{0} = \mathbb{Z} \oplus \mathbb{Z}\oplus \mathbb{Z}$


$\partial_1(a) = $ $\partial_1(b) = $ $\partial_1(a) = 0$


$\implies \bold{Ker}$ $\partial_1 = <a,b,c>$

$\implies \bold{Im}$ $\partial_1 = 0$

$\implies H_1(T^3) =
\frac{\bold{Ker}\enskip\partial_1}{\bold{Im}\enskip\partial_2} =
\frac{<a,b,c>}{0} =
= \mathbb{Z}\oplus \mathbb{Z} \oplus \mathbb{Z}$

$\partial_0(v) = 0$

$\implies \bold{Ker}$ $\partial_0 = <v$

$\implies \bold{Im}$ $\partial_0 = 0$

$\implies H_0(T^3) =
\frac{\bold{Ker}\enskip\partial_0}{\bold{Im}\enskip\partial_1} =
\frac{<v>}{0} = <v>
= \mathbb{Z}$

So,

$H_n(T^3)
= \begin{cases}\mathbb{Z}, n \in\{0,3\}\\
  \mathbb{Z}\oplus \mathbb{Z}\oplus \mathbb{Z}, n \in \{1,2\} \\
  0,$ else$
\end{cases}$



\newpage

\paragraph{12. (a)} See drawing of X.


$C_2(X) = \{\lambda f |$ $\lambda \in\mathbb{Z}\}$

$C_1(X) = \{\lambda_0a+\lambda_1b +\lambda_2c +\lambda_3d|$ $\lambda_i\in\mathbb{Z}\}$

$C_0(X) = \{\lambda v |$ $\lambda\in\mathbb{Z}\}$

$\partial_2(\lambda f) = 2\lambda(a+b+c) \equiv 0$ $($mod $2)$

$\implies \bold{Im}$ $\partial_2 = 0$

$\implies \bold{Ker}$ $\partial_2 = <f>$

$C_n(K) = 0$, for $n \geq 3 \implies \bold{Im}$ $\partial_n = 0$

$\implies H_n(X) = \frac{0}{0}$, for $n \geq 3$

$\implies H_2(X) = \frac{<f>}{0} = \mathbb{Z}_2$

$\partial_1(a) = \partial_1(b) = \partial_1(c) = \partial_1(d) = 0$

$\implies \bold{Im}$ $\partial_1 = 0$

$\implies \bold{Ker}$ $\partial_1 = <a,b,c,d>$

$\implies H_1(X) = \frac{<a,b,c,d>}{0} = \mathbb{Z}_2\oplus \mathbb{Z}_2 \oplus
\mathbb{Z}_2 \oplus \mathbb{Z}_2$

$\partial_0(v) = 0$

$\implies \bold{Ker}$ $\partial_0 = <v>$

$\implies H_0(X) = \frac{<v>}{0} = \mathbb{Z}_2$

So

$H_n(X) = \begin{cases}
  \mathbb{Z}_2\quad\quad\quad\quad\quad\quad\quad, n \in \{0,2\}\\
  \mathbb{Z}_2\oplus \mathbb{Z}_2 \oplus
  \mathbb{Z}_2 \oplus \mathbb{Z}_2\quad, n = 1\\
  0\quad\quad\quad\quad\quad\quad\quad,$ else $
\end{cases}$


(c) By The Universal Coefficient Theorem for $\mathbb{Z}_2$:

$\forall n \in \mathbb{N}: n\geq 3:$

$H_n(X;\mathbb{Z}_2) = (H_n(X)\otimes \mathbb{Z}_2)\oplus
$Tor$(H_{n-1}(X), \mathbb{Z}_2) = (0 \otimes \mathbb{Z}_2)\oplus
$Tor$(0, \mathbb{Z}_2) = 0 \oplus 0 = 0 $

$H_2(X;\mathbb{Z}_2) = (H_2(X)\otimes \mathbb{Z}_2)\oplus
$Tor$(H_1(X), \mathbb{Z}_2) = (0 \otimes \mathbb{Z}_2)\oplus
$Tor$(\mathbb{Z}\oplus \mathbb{Z}\oplus \mathbb{Z} \oplus
\mathbb{Z}_2, \mathbb{Z}_2) = 0\oplus 0\oplus 0\oplus 0\oplus\mathbb{Z}_2 = \mathbb{Z}_2 $


$H_1(\mathbb{R}P^2;\mathbb{Z}_2) = (H_1(\mathbb{R}P^2)\otimes \mathbb{Z}_2)\oplus
$Tor$(H_0(\mathbb{R}P^2), \mathbb{Z}_2) = ([\mathbb{Z}\oplus
\mathbb{Z}\oplus \mathbb{Z} \oplus \mathbb{Z}_2]\otimes \mathbb{Z}_2)\oplus
$Tor$(\mathbb{Z}, \mathbb{Z}_2)$

$ =  \mathbb{Z}\otimes \mathbb{Z}_2 \otimes
\mathbb{Z}\otimes \mathbb{Z}_2\oplus \mathbb{Z}\otimes \mathbb{Z}_2
\oplus \mathbb{Z}_2\otimes \mathbb{Z}_2 \oplus 0 = \mathbb{Z}_2\oplus
\mathbb{Z}_2\oplus \mathbb{Z}_2\oplus \mathbb{Z}_2$

$H_0(X;\mathbb{Z}_2) = (H_{0}(X)\otimes
\mathbb{Z}_2)\oplus $Tor$(H_{-1}(X), \mathbb{Z}_2) = (\mathbb{Z}\otimes
\mathbb{Z}_2)\oplus $Tor$(0, \mathbb{Z}_2) = \mathbb{Z}_2\oplus 0 = \mathbb{Z}_2$






\end{document}

%%% Local Variables:
%%% mode: latex
%%% TeX-master: t
%%% End:
