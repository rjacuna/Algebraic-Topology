\documentclass{article}
\usepackage{fontspec}
\usepackage{xcolor}

\usepackage{amsthm}
\usepackage{amsmath}
\usepackage{amssymb}
\usepackage{unicode-math}
\usepackage[makeroom]{cancel}

\usepackage[normalem]{ulem}

\setmainfont{Times New Roman}
\setmathfont{Latin Modern Math}

\setlength\parindent{0em}
\setlength\parskip{0.618em}
\usepackage[a4paper,lmargin=1in,rmargin=1in,tmargin=1in,bmargin=1in]{geometry}

\usepackage{enumitem}
\renewcommand\qed{$\blacksquare$}

\usepackage{mathabx,graphicx}
\usepackage{wrapfig}

\begin{document}

\begin{center}
  \textbf{MATH} 145B---\textbf{HOMEWORK} 4

  \color{red}R\color{teal}icardo
  \color{red}J\color{cyan}.
  \color{red}A\color{teal}cu$\color{red}{\widetilde{
      \color{teal}\text{n}}}$\color{teal}a\color{black}

  \color{teal}(\color{red}862079740\color{teal})\color{black}
\end{center}\vspace{1.618em}

\color{red}NOTE: All functions under discussion are considered
continuous, unless that's a property to be proved.\color{black}

\subsubsection*{6.4}

\paragraph{3} Let $f: S^1\rightarrow S^1$ be a continuous mapping such
that deg$(f) \neq 1$. Prove that $f$ has a fixed point.

\uwave{Pf\enskip. }
\vspace{0.618 em}

Ass. deg$(f) \neq 1$\quad $\guillemotright \triangle \guillemotleft$

Let $S^1:= \{z\in \mathbb{C}|\quad |z| = 1\}$

Suppose $f$ doesn't have a fixed point---i.e. $\forall z \in
\mathbb{C}: f(z)\neq z$\quad $\guillemotright \lozenge \guillemotleft$

Consider $H: S^1\times [0,1]\rightarrow S^1; (z,t) \mapsto \frac{(1
  -t)f(z) -tz}{||(1 -t)f(z) -tz||}$

Since $H$ is a rational multi-variable function, $H$ is continuous iff
it's denominator is never $0$. And, $||x||$ is $0$ iff $x = 0$. So we
only need to check that case.

$(1 -t)f(z) -tz \stackrel{?}{=} 0$

$\implies (1 -t)f(z) = tz$\quad $\guillemotright * \guillemotleft$

$\stackrel{||.||}{\implies} ||(1 -t)f(z)|| =  ||tz||$

$\implies |(1 -t)|\cdot||f(z)|| =  |t|\cdot||z||$

$f(z)$, $z$ $\in S^1$ $\implies ||f(z)|| = ||z|| =1$

$\implies |(1 -t)|\cdot 1 =  |t|\cdot 1$

$t<1 \implies 1 -t = |t|$

$\implies 1-t = t$

$\implies 1 = 2t \implies t
= \frac{1}{2}$

$\guillemotright * \guillemotleft$ $\implies \frac{1}{2}f(z) =
\frac{1}{2}z$

$\stackrel{\cdot 2}{\implies} f(z) = z$\quad $\guillemotright \blacklozenge \guillemotleft$

$\guillemotright \lozenge \guillemotleft$ $\implies H$ is continuous
since $\guillemotright \blacklozenge \guillemotleft$ can't happen.

$H(z,0) = \frac{f(z)}{||f(z)||} = \frac{f(z)}{1} = f(z)$

$H(z,1) = \frac{z}{||z||} = \frac{z}{1} = z = 1_{s^1}(z)$

$\implies f\sim 1_{S^1}$

$\implies$ deg$(f) = $ deg$(1_{S^1}) = 1$

This contradicts\quad $\guillemotright \triangle \guillemotleft$

$\implies$ $\exists z_0\in \mathbb{C}: f(z_0) = z_0$

So, $f$ has a fixed point.

\vspace{0.618  em}
$\blacksquare$

\newpage
\paragraph{6} Show that if $A$ is a nonsingular $3$ by $3$ matrix having nonnegative entries, then $A$ has
a positive real eigenvalue.

\uwave{Pf\enskip. }
\vspace{0.618 em}

Let $A$ be a nonsingular $3$ by $3$ matrix having nonnegative entries.

$\implies$ The entries of $A$ are positive real numbers, because
there's no such thing as a complex negative numbers.

Denote the set of all $3$ by $3$ real valued  matrices by
Mat$_{3\times3}(\mathbb{R})$.

And, the set of
all linear endomorphisms of $\mathbb{R}^3$ by $\mathcal{L}(\mathbb{R}^3)$.

Mat$_{3\times3}(\mathbb{R})$$\simeq$ $\mathcal{L}(\mathbb{R}^3)$ (by
Linear Algebra)

$\implies \exists\enskip (\enskip T:
\mathbb{R}^3$\enskip\rotatebox{90}{$\circlearrowleft$}\enskip $)$ such that $A$
is the matrix representation of $T$

$S^2_{\geq 0}:= (\mathbb{R}^2\times (\{0\}\cup \mathbb{R^+})) \cap
S^2$

Consider $T^*: \mathbb{R}^3\rightarrow S^2_{\geq
  0}; (x,y,z)
\mapsto \frac{T((x,y,z))}{||T((x,y,z))||}$

$T^{*}$ is continuouse since $||T((x,y,z))|| = 0$ iff $T((x,y,z)) =
(0,0,0)$ since, $T$ is linear $T((x,y,z)) = 0$ iff $(x,y,z) =
(0,0,0)$. So $||T((x,y,z))|| = 0 \implies (0,0,0)/0 = (0,0,0)$. So
$T^{*}$ is well defined.

Consider $T^{*}|_{S^2_{\geq 0}}: S^2_{\geq
  0}$\enskip\rotatebox{90}{$\circlearrowleft$}\enskip, the restriction
of $T^{*}$ to $S^2_{\geq 0}$


Since the image of the stereographic projection of $S^2_{\geq 0}$ is
$D^2$, and the stereographic projection admits a continuous inverse. It
follows that $S^2_{\geq 0}\simeq D^2$.


$\implies \exists (f: D^2
\rightarrow S^2_{\geq 0})$
and $(g: S^2_{\geq 0}\rightarrow D^2):$
$f\circ g =  1_{D^2}$ and $g\circ f = 1_{S^3_{\geq 0}$

$\implies g\circ T^{*}|_{S^2_{\geq 0}}\circ f: D^2\rightarrow D^2$

$f, g, T^{*}|_{S^2_{\geq 0}}$ are continuous $\implies g\circ T^{*}|_{S^2_{\geq 0}}\circ f$ is continuous

$\implies g\circ T^{*}|_{S^2_{\geq 0}} \circ f$ has a fixed point (by Brawer's)

$\implies \exists (x,y) \in D^2: g\circ T^{*}|_{S^2_{\geq 0}} \circ f((x,y)) = (x,y)$

$\stackrel{f}{\implies} f \circ g\circ T^{*}|_{S^2_{\geq 0}} \circ f((x,y)) = f((x,y))
\in S^2_{\geq 0}$

$\implies 1_{S^2_{\geq 0}}\circ T^{*}|_{S^2_{\geq 0}} \circ f((x,y)) = f((x,y))$

$\implies T^{*}|_{S^2_{\geq 0}} \circ f((x,y)) = f((x,y))$

$\implies T^{*}|_{S^2_{\geq 0}}(f((x,y))) = \frac{T(f(x,y))}{||T(f(x,y))||} =
f((x,y))$

$\implies \frac{T(f(x,y))}{||T(f(x,y))||} = f((x,y))$

$\implies T(f(x,y)) = ||T(f(x,y))|| f((x,y))$

Let $\lambda := ||T(f(x,y))|| \in \mathbb{R}$
and $(a,b,c) := f(x,y) \in S^2_{\geq 0} \implies (a,b,c) \in
R^3$

$\implies A(a,b,c)^T = \lambda (a, b, c)^T$ (by the one-to-one
correspondence between $A$ and $T$)

$\implies A$ has a positive real eigenvalue

\vspace{0.618  em}
$\blacksquare$


\subsection*{7.2}
\paragraph{1}  Give an example of space build of two vertices and two
edges which is homeomorphic to the simplicial circle but not a simplicial complex.


\begin{wrapfigure}{r}{0.5\textwidth}
  \includegraphics[scale=0.15]{dots} \end{wrapfigure}
If you normalize
the edges you get a circle. However, it is not a simplicial complex,
because the intersection of the top and bottom lines is two
disconnected points. Which isn't a simplex. So, it isn't a simplicial complex.

\paragraph{3}

(a)  Draw a triangulation of the cylinder.

  \includegraphics[scale=0.25]{cylinder}

  (b)  Calculate the Euler number of the cylinder.

  $\chi\left(  \includegraphics[scale=0.10,
    angle=66,origin=c]{cylinder}\right) = 6 - 12 + 6 = 0$

\paragraph{7} a)  Draw a triangulation of the surface of genus two,
$Σ_2$.

\includegraphics[scale=0.2]{g2}


(b)  Calculate the Euler number of the surface of genus two, $Σ_2$.

$\chi\left(  \includegraphics[scale=0.10,
    angle=66,origin=c]{g2}\right) = 18 - 64 + 44 = -2$


\end{document}

%%% Local Variables:
%%% mode: latex
%%% TeX-master: t
%%% End:
