\documentclass{article}
\usepackage{fontspec}
\usepackage{xcolor}

\usepackage{amsthm}
\usepackage{amsmath}
\usepackage{amssymb}
\usepackage{unicode-math}
\usepackage[makeroom]{cancel}

\usepackage[normalem]{ulem}

\setmainfont{Times New Roman}
\setmathfont{Latin Modern Math}

\setlength\parindent{0em}
\setlength\parskip{0.618em}
\usepackage[a4paper,lmargin=1in,rmargin=1in,tmargin=1in,bmargin=1in]{geometry}

\usepackage{enumitem}
\renewcommand\qedsymbol{$\blacksquare$}

\begin{document}

\begin{center}
  \textbf{MATH} 145B---\textbf{HOMEWORK} 3

  \color{red}R\color{teal}icardo
  \color{red}J\color{cyan}.
  \color{red}A\color{teal}cu$\color{red}{\widetilde{\color{teal}\text{n}}}$\color{teal}a\color{black}

  \color{teal}(\color{red}862079740\color{teal})\color{black}
\end{center}\vspace{1.618em}

\color{red}NOTE: All functions under discussion are considered
continuous, unless that's a property to be proved.\color{black}

\subsubsection*{6.3}

\paragraph{1} Each of the spaces below is either contractible or homotopy equivalent to $S^1$ or neither.
For each example, determine which alternative holds. You do not need to give detailed
proofs but please give a short explanations. \color{red} Note:  I know
I don't have to prove any of this, but I did it already so feel free
to skip to the conclusion. \color{black}

(a) The solid torus $D^2 × S^1$

\uwave{Pf\enskip.}
\vspace{0.309 em}

 $D^2 × S^1 =\\ \{(x,y)\in \mathbb{R}^2|\quad ||(x,y)|| \leq 1\}\times
 \{(x,y)\in \mathbb{R}|\quad ||(x,y)|| = 1 \}=\\
 \{[(x,y),(z,w)] \in \mathbb{R}^2\times\mathbb{R}^2|\quad  ||(x,y)||
 \leq 1$ and $||(z,w)|| = 1\}$

 $f: D^2 × S^1 \rightarrow S^1; [(x,y),(z,w)] \mapsto (z,w)$
 and
 $g: S^1 \rightarrow D^2 × S^1; (z,w) \mapsto [(0,0),(z,w)]$

 $f\circ g ((z,w)) = f(g((z,w))) = f([(0,0),(z,w)]) = (z,w)$
 $\implies f\circ g = 1_{S^1}$


 $g\circ f([(x,y),(z,w)]) = g(f([(x,y),(z,w)])) = g((z,w)) =
 [(0,0),(z,w)]$

 Consider $H: (D^2 × S^1)\times [0,1] \rightarrow D^2\times S^1;
 ([(x,y),(z,w)],t) \mapsto [t(x,y),(z,w)]$

$H$ is continuous since the first coordinate of the output is the
multiplication of two continuous functions, and the second coordinate
is the same.

$H([(x,y),(z,w)],1) = [(x,y),(z,w)] \implies H([(x,y),(z,w)],1) =
1_{D^2\times S^1}$\\
$H([(x,y),(z,w)],0) = [(0,0),(z,w)] \implies H([(x,y),(z,w)], 0) =
g\circ f$

$\implies g\circ f \sim 1_{D^2\times S^1}$\\

$D^2 × S^1 \sim S^1$\\



\vspace{0.309 em}
$\blacksquare$

(c) The cylinder $S^1 × \mathbb{R}$

\uwave{Pf\enskip.}
\vspace{0.618 em}

$S^1 × \mathbb{R} = \{(x,y)\in \mathbb{R}^2|\quad ||(x,y)|| = 1\}\times
\mathbb{R} = \{[(x,y), z] \in \mathbb{R}^2\times \mathbb{R}| ||(x,y)||
= 1\}$

 $f: S^1 × \mathbb{R} \rightarrow S^1; [(x,y),z] \mapsto (x,y)$
 and
 $g: S^1 \rightarrow S^1 × \mathbb{R}; (x,y) \mapsto [(x,y), 0]$

 $f\circ g ((x,y)) = f(g((x,y))) = f([(x,y),0]) = (x,y)$
 $\implies f\circ g = 1_{S^1}$

 $g\circ f([(x,y),z]) = g(f([(x,y),z])) = g((x,y)) =
 [(x,y),0]$


 Consider $H: (S^1 × \mathbb{R})\times [0,1] \rightarrow S^1 × \mathbb{R};
 ([(x,y),z],t) \mapsto [(x,y),tz]$

$H$ is continuous since the second coordinate of the output is the
multiplication of two continuous functions, and the first coordinate
is the same.

$H([(x,y),z],1) = [(x,y),z] \implies H([(x,y),z],1) =
 1_{S^1 × \mathbb{R}}$\\
$H([(x,y),z],0) = [(x,y),0] \implies H([(x,y),z], 0) =
g\circ f$

$\implies g\circ f \sim 1_{S^1 × \mathbb{R}}
\implies$
$S^1 × \mathbb{R} \sim S^1$


\vspace{0.618 em}
$\blacksquare$


(e) The set of all points $(x, y) ∈ \mathbb{R}^2$ such
that $||(x, y)|| > 1$\\

\uwave{Pf\enskip. }
\vspace{0.618 em}

$X$ $:= \{(x, y) ∈ \mathbb{R}^2|$ $||(x, y)|| > 1\}$

$f: X \rightarrow S^1; (x,y) \mapsto \frac{(x,y)}{||(x,y)||}$
and
$g: S^1 \rightarrow X; (x,y) \mapsto 2(x,y)$

$f\circ g ((x,y)) = f(g((x,y))) = f(2(x,y)) =
\frac{2(x,y)}{||2(x,y)||} = \frac{2(x,y)}{2||(x,y)||} =
\frac{(x,y)}{||(x,y)||}$

Since, $(x,y) \in s^1$, $||(x,y)||=1$. So, $f\circ g ((x,y)) = (x,y)$

$\implies f\circ g = 1_{s^1}$

$g\circ f ((x,y)) = g(f((x,y))) = g(\frac{(x,y)}{||(x,y)||}) =
2\frac{(x,y)}{||(x,y)||}$

Consider $H: X\times [0,1]\rightarrow X; [(x,y),t] \mapsto \left( \frac{t||(x,y)||
    +2(1-t)}{||(x,y)||}\right)(x,y)$

$H$ is continuous since it's a rational function of three
variables times a vector, and the denominator is never $0$ as $(x,y) \in X$.

$H((x,y),0) = \left( \frac{0||(x,y)||
    +2(1-0)}{||(x,y)||}\right)(x,y) =
2\frac{(x,y)}{||(x,y)||}$ $\implies H((x,y),0) = g\circ f$

$H((x,y),1) = \left( \frac{1||(x,y)||
    +2(1-1)}{||(x,y)||}\right)(x,y) = \frac{||(x,y)||}{||(x,y)||}(x,y)
=(x,y)$ $\implies H((x,y),1) = 1_X$

$\implies g\circ f \sim 1_x$

$\implies X \sim S^1$



\vspace{0.618 em}
$\blacksquare$
\newpage
(g)  The subset of $\mathbb{R}^2$ given by $S^1 \cup (\mathbb{R}^+ × \{0\})$.\\

\uwave{Pf\enskip. }
\vspace{0.618 em}

$f: S^1 \cup (\mathbb{R}^+ × \{0\}) \rightarrow S^1$ defined by\\
$f((x,y)) = \begin{cases} (x,y),$ if $ (x,y) \in S^1\\ (1,0),$ if $(x,y)
  \in (\mathbb{R}^+ × \{0\}) \end{cases}$

By gluing lemma, $f$ is continuous.\\
Since, $(1,0)$ is both in $S^1$ and
$(\mathbb{R}^+ × \{0\})$.\\
And, $f((1,0))=1_{S^1}((1,0)) =(1,0)$.\\
And, $f$ is the constant map $(1,0)$ on $(\mathbb{R}^+ ×
\{0\})\backslash\{(1,0)\}$.\\
And, $f$ is the identity map on
$S^1\backslash\{(1,0)\}$.

and $g: S^1 \hookrightarrow S^1 \cup (\mathbb{R}^+ × \{0\});
(x,y)\mapsto (x,y)$

$f \circ g((x,y)) = f(g((x,y))) = f((x,y)) = \begin{cases} (x,y),$ if $ (x,y) \in S^1\\ (1,0),$ if $(x,y)
  \in (\mathbb{R}^+ × \{0\}) \end{cases}$

Since the domain of $g$ is $S^1$ $f\circ g((x,y)) = (x,y) \implies
f\circ g = 1_{s^1}$

$g \circ f((x,y)) = g(f(x,y))\\ = g\left(\begin{cases} (x,y),$ if $ (x,y) \in S^1\\ (1,0),$ if $(x,y)
    \in \mathbb{R}^+ × \{0\} \end{cases}\right)\\ = \begin{cases} g((x,y)),$ if $ (x,y) \in S^1\\ g((1,0)),$ if $(x,y)
  \in \mathbb{R}^+ × \{0\} \end{cases}\\ = \begin{cases} (x,y),$ if $ (x,y) \in S^1\\ (1,0),$ if $(x,y)
  \in \mathbb{R}^+ × \{0\} \end{cases}$

Consider $H: (S^1 \cup (\mathbb{R}^+ × \{0\}))\times [0,1] \rightarrow
S^1 \cup (\mathbb{R}^+ × \{0\})$ defined by

$H((x,y),t) = \begin{cases}
  (x,y),$ if $(x,y) \in S^1\\
  t(1,0) +(1-t)(x,0),$ if $(x,y) \in \mathbb{R}^+ × \{0\}$ $\end{cases}$

$H((x,y),0) = \begin{cases}
  (x,y),$ if $(x,y) \in S^1\\
  0(1,0) +(1-0)(x,0),$ if $(x,y) \in \mathbb{R}^+ × \{0\}$
  $\end{cases} = \begin{cases}
  (x,y),$ if $(x,y) \in S^1\\
  (x,0),$ if $(x,y) \in \mathbb{R}^+ × \{0\}$ $\end{cases}$

$\implies H((x,y),0) = 1_{S^1 \cup (\mathbb{R}^+ × \{0\})}$

$H((x,y),1) = \begin{cases}
  (x,y),$ if $(x,y) \in S^1\\
  1(1,0) +(1-1)(x,0),$ if $(x,y) \in \mathbb{R}^+ × \{0\}$
  $\end{cases} = \begin{cases}
  (x,y),$ if $(x,y) \in S^1\\
  (1,0),$ if $(x,y) \in \mathbb{R}^+ × \{0\}$ $\end{cases}$

$\implies H((x,y),1) = f\circ g$

$H$ is continuous when restricted to $(\mathbb{R}^+ ×
\{0\})\times[0,1]$ since the restricted map is a straight line
homotopy. $H$ is also continuous when restricted to
$S^1\times[0,1]$ since it's the identity homotopy. Since $((1,0),t)$ is
both on $(\mathbb{R}^+ ×
\{0\})\times[0,1]$ and $S^1\times[0,1]$, we only need to show both cases
agree for all $t$.

$H((1,0),t) = \begin{cases}
  (1,0),$ if $(1,0) \in S^1\\
  t(1,0) +(1-t)(1,0),$ if $(1,0) \in \mathbb{R}^+ × \{0\}$
  $\end{cases}\\ = \begin{cases}
  (1,0),$ if $(1,0) \in S^1\\
  t(1,0) +1(1,0) -t(1,0),$ if $(1,0) \in \mathbb{R}^+ × \{0\}$
  $\end{cases}\\ = \begin{cases}
  (1,0),$ if $(1,0) \in S^1\\
  (t-t)(1,0) + (1,0),$ if $(1,0) \in \mathbb{R}^+ × \{0\}$
  $\end{cases} =
\begin{cases}
  (1,0),$ if $(1,0) \in S^1\\
  (1,0),$ if $(1,0) \in \mathbb{R}^+ × \{0\}$ $\end{cases} = (1,0)$

$\implies g\circ f \sim 1_{S^1 \cup (\mathbb{R}^+ × \{0\})} \implies
S^1 \cup (\mathbb{R}^+ × \{0\}) \sim S^1$

\vspace{0.618  em}
$\blacksquare$

(i)  The subset of $\mathbb{R}^2$ given by $S^1 \cup (\mathbb{R}^+ × \mathbb{R})$

\uwave{Pf\enskip. }
\vspace{0.618 em}


Note, $(0,0) \notin$ $s^1 \cup (\mathbb{R}^+ × \mathbb{R})$\\
(This observation is implicit because, the question (h)
considered $s^1 \cup ([0,+\infty) × \mathbb{R})$)

$f: s^1 \cup (\mathbb{R}^+ × \mathbb{R}) \rightarrow S^1; \frac{(x,y)}{||(x,y)||}$
and $g: S^1 \hookrightarrow S^1 \cup (\mathbb{R}^+ × \mathbb{R});
(x,y)\mapsto (x,y)$

$f\circ g ((x,y)) = f(g((x,y))) = f((x,y))= \frac{(x,y)}{||(x,y)||}$

Since the domain of $g$ is $S^1$ $(x,y) \in S^1 \implies ||(x,y)|| =
1$

$\implies f\circ g ((x,y)) = (x,y) \implies f\circ g = 1_{S^1}$

$g\circ f((x,y)) = g(f(x,y)) = g(\frac{(x,y)}{||(x,y)||}) =\frac{(x,y)}{||(x,y)||} $

Consider $H: s^1 \cup (\mathbb{R}^+ × \mathbb{R}) \times [0,1]\rightarrow s^1 \cup (\mathbb{R}^+ × \mathbb{R}); [(x,y),t] \mapsto \left( \frac{t||(x,y)||
    +(1-t)}{||(x,y)||}\right)(x,y)$

$H$ is continuous since it's a rational function of three
variables times a vector, and the denominator is never $0$ as $(x,y)
\in s^1 \cup (\mathbb{R}^+ × \mathbb{R})$.

$H((x,y),0) = \left( \frac{0||(x,y)||
    +(1-0)}{||(x,y)||}\right)(x,y) =
\frac{(x,y)}{||(x,y)||}$ $\implies H((x,y),0) = g\circ f$

$H((x,y),1) = \left( \frac{1||(x,y)||
    +(1-1)}{||(x,y)||}\right)(x,y) = \frac{||(x,y)||}{||(x,y)||}(x,y)
=(x,y)$ $\implies H((x,y),1) = 1_{s^1 \cup (\mathbb{R}^+ × \mathbb{R})}$

$\implies g\circ f \sim 1_{s^1 \cup (\mathbb{R}^+ × \mathbb{R})}$

$\implies s^1 \cup (\mathbb{R}^+ × \mathbb{R}) \sim S^1$

\vspace{0.618  em}
$\blacksquare$

The, proof of (g) is ok, but it's somewhat involved as I assumed
$(0,0)$ was in $S^1 \cup (\mathbb{R}^+ × \{0\})$. I'm convinced now it
is not the case, and in fact one can replace $\{0\}$ for $\mathbb{R}$
in the previous proof and it proves (g) much easier.
\newpage

\paragraph{2}

(a)  Let $f, g:S^1→S^1$ be continuous mappings and let’s take the
complex multiplication operation on $S^1 ⊂ \mathbb{C}$.  Define $h(z)$
to be the product $h(z) =f(z)·g(z)$.  Show that deg$(h)$ is equal to
deg$(f) +$ deg$(g)$

\uwave{Pf\enskip. }
\vspace{0.618 em}

$f: S^1\rightarrow S^1$ $\implies \exists n\in \mathbb{Z}: f \sim z^n$

$\implies \exists (F: S^1\times [0,1]\rightarrow S^1): F(z,1) = f(z)$
and $F(s,0) = z^n$

$g: S^1\rightarrow S^1$ $\implies \exists m\in \mathbb{Z}: g \sim z^m$

$\implies \exists (G: S^1\times [0,1]\rightarrow S^1): G(z,1) = g(z)$
and $G(z,0) = z^m''

$``\cdot''$ denotes the complex multiplication operation on $\mathbb{C}$

$h: S^1\rightarrow S^1; z \mapsto f(z)\cdot g(z)$

Consider $H: S^1\times [0,1]\rightarrow S^1$ defined by

$H(z,t) = F(z,t)\cdot G(z,t)$

$H$ is continuous since it's the complex multiplication of two
continuous functions.

$H(z,0) = F(z,0)\cdot G(z,0) = z^n\cdot z^m$

and

$H(z,0) = F(z,1)\cdot G(z,1) = f(z)\cdot g(z) = h(z)$

$\implies h \sim z^n\cdot z^m = z^{n+m}$

$\implies $deg$(h) =$ deg$(z^{n+m}) = n + m = $ deg$(f) +$ deg$(g)$

\vspace{0.618  em}
$\blacksquare$

(b)  If $f, g:S^1→S^1$ are homotopic continuous mappings, then deg$(f)
= $ deg$(g)$

\uwave{Pf\enskip. }
\vspace{0.618 em}

$f, g:S^1→S^1$ are homotopic continuous mappings

$\implies$ $\exists n \in \mathbb{Z}: f\sim z^n$ and
$\exists m \in \mathbb{Z}: g\sim z^m$

$\implies$ deg$(f) = n$ and deg$(g) = m$

$f\sim g$ and $g\sim z^m \implies f\sim z^m$

$\implies$ deg$(f)= m$

$\implies$ deg$(f) =$ deg$(g)$

\vspace{0.618  em}
$\blacksquare$


\newpage

\paragraph{3}  If $f, g:S^1→S^1$ are two continuous maps, express
deg$(f\circ g)$ in terms of deg$(f)$ and deg$(g)$.  Use this to show that
$f\circ g$ is homotopic to $g\circ f$.

\uwave{Pf\enskip. }
\vspace{0.618 em}

$f: S^1\rightarrow S^1$ $\implies \exists n\in \mathbb{Z}: f \sim z^n$

$\implies \exists (F: S^1\times [0,1]\rightarrow S^1): F(z,1) = f(z)$
and $F(s,0) = z^n$

$g: S^1\rightarrow S^1$ $\implies \exists m\in \mathbb{Z}: g \sim z^m$

$\implies \exists (G: S^1\times [0,1]\rightarrow S^1): G(z,1) = g(z)$
and $G(z,0) = z^m$

Consider $H: S^1\times [0,1]\rightarrow S^1$ defined by

$H(z,t) = F(G(z,t),t)$

$H$ is continuous since it's the composition of two
continuous functions.

$H(z,1) = F(G(z,1),1) = F(g(z),1) = f(g(z)) = f\circ g (z)$

and

$H(z,0) = F(G(z,0),0) = F(z^m,1) = (z^m)^n = z^{mn}$

$\implies f\circ g \sim z^{nm}$

$\implies $deg$(f\circ g) =$ deg$(z^{nm}) = nm = $ deg$(f)$ deg$(g)$

By the symmetry, one can evaluate the previous sentence for $f = g$
and $g = f$, so it becomes:

deg$(g\circ f) =$ deg$(z^{mn}) = mn = $ deg$(g)$ deg$(f)$ $=$ deg$(f)$ deg$(g)$

$\implies$ deg$(f\circ g) =$ deg$(g\circ f)$

$\implies$ $f\circ g \sim g\circ f$ (by 35 Corollary in Circle Lecture
Notes)

\vspace{0.618  em}
$\blacksquare$


\end{document}

%%% Local Variables:
%%% mode: latex
%%% TeX-master: t
%%% End:
